\documentclass[a4paper,10pt,reqno,nonamelimits]{article}
%\usepackage[pagebackref,colorlinks=true,urlcolor=blue,linkcolor=blue,citecolor=blue]{hyperref}
\usepackage[colorlinks=true,urlcolor=blue,linkcolor=blue,citecolor=blue]{hyperref}
%\usepackage[colorlinks=false]{hyperref}

%\usepackage[top=2cm,bottom=2cm,left=1.5cm,right=7cm,marginparwidth=6.5cm]{geometry}

%% \usepackage{makeidx}
%%\usepackage{theorem}
%%\usepackage{array}

%\usepackage{mathptmx}        % PSNFSS
%\usepackage{bookman}         % PSNFSS (wide!)
\usepackage{newcent}         % PSNFSS

\usepackage{amsmath}
\usepackage{amssymb}
\usepackage{amsfonts}
\usepackage{amscd}
\usepackage{amsthm}
\usepackage{amsxtra}
%\usepackage{stmaryrd}
%\usepackage{mathabx}
%\usepackage{eucal}
\usepackage{mathrsfs}
\usepackage{nicefrac}\newcommand{\nfrac}{\nicefrac}
\usepackage{bbold}
%%\usepackage{bm}
\usepackage{graphicx}%\graphicspath{ {image/} }
\usepackage{xcolor}
% \usepackage[raggedright,IT,hang]{subfigure}
%\usepackage{url}
\usepackage{tikz}  \usetikzlibrary{quantikz}
%\usepackage{wrapfig}
\usepackage{enumitem}
% \setlist[enumerate]{leftmargin=0em,itemindent=0em, labelindent=0pt,labelwidth=1.5em,labelsep=.5em, align=left}
% \newlist{txtenum}{enumerate}{1}
% \setlist[txtenum]{leftmargin=0em,itemindent=0em, labelindent=0pt,labelwidth=1em,labelsep=2ex, align=left}
% \begin{enumerate}[resume,label=(\alph*)]...
% \begin{enumerate}[label=\arabic*., leftmargin=1em,itemindent=.5em, labelindent=0pt,labelwidth=1em,labelsep=.5em, align=left]
% * H spacing: leftmargin+itemindent = labelindent+labelwidth+labelsep
% * Sub-numbers: just take ``label*'' instead of ``label''

\usepackage[norelsize,boxed,noend,linesnumbered]{algorithm2e}\DontPrintSemicolon
%\SetKwFor{RepeatTimes}{repeat}{times}{endrepeat}
%\RestyleAlgo{ruled}
% 'norelsize' takes care of incompatibility with amsart in old version of algorithm2e.
%             Remove it once the new version is on the system
% \begin{algorithm}[htbp OR H]
%   Commands: \TitleOfAlgo{}
\SetKwInOut{Input}{Input}\SetKwInOut{Output}{Output}\SetKwInOut{Oracle}{Oracle}
% \For{}{} \Foreach{}{} \If{}{} \eIf{}{}{} \EsleIf{}{} \While{}{} \Return{}

\usepackage{datetime}

\newcounter{mysaveenumi}
% U S A G E :
% \setcounter{mysaveenumi}{\theenumi}
% \begin{enumerate}[(a)]\setcounter{enumi}{\themysaveenumi}
%   ...

%\newtheorem{fact}[theorem]{Fact}
% \numberwithin{theorem}{section}

% \renewcommand{\thesubsection}{\thesection.\alph{subsection}}

% \newcommand{\myparagraph}[1]{\mypar #1}
% \newcommand{\myparagraphwskip}[1]{\smallskip\paragraph{#1}}
% \newcommand{\claimpf}[1]{\myparagraph{\textit{#1}}}
% %


%%%%%%%%%%%%%%%%%%%%%%%%%%%%%%%%%%%%%%%%%%%%%%%%%%%%%%%%%%%%%%%%%%%%%%%%%%%%%%%%%%%%%%%%%%%%%%%%%%%%

\newcommand{\mypar}{\par\medskip\noindent}
\newcommand{\myparbig}{\par\bigskip\noindent}

\newcommand{\rip}[2]{\langle #1 \mid #2 \rangle}
\newcommand{\cip}[2]{( #1 \mid #2 )}

\newcommand{\Nm}[1]{\left\| #1 \right\|} % big norm
\newcommand{\nm}[1]{\lVert #1 \rVert}    % small norm

\DeclareMathOperator{\Span}{Span}
\DeclareMathOperator{\Ker}{Ker}
\DeclareMathOperator{\Img}{Img}
\DeclareMathOperator{\tr}{tr}

\newcommand{\RR}{\mathbb R}
\newcommand{\CC}{\mathbb C}
\newcommand{\KK}{\mathbb K}

\renewcommand{\AA}{\mathbb A}
\newcommand{\BB}{\mathbb B}
\newcommand{\DD}{\mathbb D}
\newcommand{\HH}{\mathbb H}

\newcommand{\NN}{\mathbb N}
\newcommand{\ZZ}{\mathbb Z}

\newcommand{\LL}{\mathbb L}
\newcommand{\MM}{\mathbb M}

\newcommand{\PP}{\mathbb P}

\newcommand{\cA}{\mathcal A}
\newcommand{\cB}{\mathcal B}
\newcommand{\cC}{\mathcal C}
\newcommand{\cH}{\mathcal H}
\newcommand{\cG}{\mathcal G}

\newcommand{\One}{\mathbb{1}}
\newcommand{\one}{\mathbf{1}}

\newcommand{\zero}{\mathbf 0}

\newcommand{\orthogonal}{{\mathrel\bot}}

\newcommand{\ANSWER}[1]{\marginpar{#1}}


\newcommand{\HWNR}{1}
\newcommand{\HandOut}{Tue Feb.~27}
\newcommand{\DueDate}{Tue March~5, 10:00}

%%%%%%%%%%%%%%%%%%%%%%%%%%%%%%%%%%%%%%%%%%%%%%%%%%%%%%%%%%%%%%%%%%%%%%%%%%%%%%%%%%%%%%%%%%%%%%%%%%%%
\begin{document}
\title{\large MTAT.05.024 Quantum Crypto\\[2ex]
  {\tiny
    Assoc.~Prof.~Dirk Oliver Theis\\
    Shahla Novruzova
  }\\[2ex]
  \LARGE Homework \# \HWNR}
%%%%%%%%%%%%%%%%%%%%%%%%%%%%%%%%%%%%%%%%%%%%%%%%%%%%%%%%%%%%%%%%%%%%%%%%%%%%%%%%%%%%%%%%%%%%%%%%%%%%%%%%%%%%%%%%%%%%%%%%%%%%%%%%%%%%
%%
%%
\date{%
  \begin{tabular}{rl}
    Handed out:                 &\HandOut                                           \\[1ex]
    Due:                        &\DueDate                                           \\
    {\small As PDF by email to} &{\small\texttt{shahla.novruzova@ut.ee}}            \\
    {\small\quad subject:}      &{\small\texttt{QCRY-HW\HWNR-}\textit{lastname}}
  \end{tabular}
}
\maketitle

\thispagestyle{empty}

\section{Orthonormal bases}\label{bas}

Definition.  The ``computational (orthonormal) basis'' of the Hilbert space of an $n$-qubit quantum register
consists of the states $\ket{x}$, where~$x$ ranges over all elements of $\{0,1\}^n$, i.e., length-$n$ bit strings.
For example, for a single qubit, we get the familiar ONB $\ket0,\ket1$.

\begin{enumerate}[label=(\alph*)]
\item\label{bell} Verify that the following four 2-qubit states form an ONB:
  \begin{itemize}
  \item $( \ket{00}+\ket{11} )/\sqrt2$

    \begin{quotation}\color{blue}
      (Your solution here.)
    \end{quotation}

  \item $( \ket{00}-\ket{11} )/\sqrt2$

    \begin{quotation}\color{blue}
      (Your solution here.)
    \end{quotation}

  \item $( \ket{01}+\ket{10} )/\sqrt2$

    \begin{quotation}\color{blue}
      (Your solution here.)
    \end{quotation}

  \item $( \ket{01}-\ket{10} )/\sqrt2$

    \begin{quotation}\color{blue}
      (Your solution here.)
    \end{quotation}

  \end{itemize}
\item Write the following 2-qubit states as superposition of the basis defined in the~\ref{bell}:
  \begin{itemize}
  \item $\ket{00}$
    \begin{quotation}\color{blue}
      (Your solution here.)
    \end{quotation}
  \item $\ket{01}$
    \begin{quotation}\color{blue}
      (Your solution here.)
    \end{quotation}
  \item $\ket{11}$
    \begin{quotation}\color{blue}
      (Your solution here.)
    \end{quotation}
  \end{itemize}
\item ............................
\end{enumerate}

\section{Measurement I}

A computational basis measurement of a single qubit has possible outcomes $0,1$.  If the single qubit is in state
\begin{equation*}
  \psi = \alpha_0\ket{0} + \alpha_1 \ket{1},
\end{equation*}
then the probability of outcome~$0$ is $|\alpha_0|^2$ and the probability of outcome~$1$ is is
$|\alpha_1|^2 = 1 - |\alpha_0|^2$.

For each of the following states, give the measurement probabilities of the outcomes:

\begin{enumerate}[label=(\alph*)]
\item $\ket0$ \ANSWER{.......}
\item $\ket1$ \ANSWER{.......}
\item $\ket{+}$ \ANSWER{.......}
\item $\ket{-}$ \ANSWER{.......}
\item $\ket{\circlearrowleft}$ \ANSWER{.......}
\item $\ket{\circlearrowright}$ \ANSWER{.......}
\end{enumerate}

\section{Measurement II}

A computational basis measurement of the \emph{left} one of two qubits has possible outcomes $0,1$.  If the two
qubits are in the state
\begin{equation*}
  \psi =  \alpha_{00}\ket{00}
        + \alpha_{01}\ket{01}
        + \alpha_{10}\ket{10}
        + \alpha_{11}\ket{11},
\end{equation*}
then the probability of outcome~$0$ is
\begin{equation*}
  |\alpha_{00}|^2 + |\alpha_{01}|^2
\end{equation*}
and the probability of outcome~$1$ is is
\begin{equation*}
  |\alpha_{10}|^2 + |\alpha_{11}|^2 = 1 - \left(  |\alpha_{00}|^2 + |\alpha_{01}|^2  \right).
\end{equation*}

For each of states in item~\ref{bell} of exercise~\ref{bas}, give the measurement probabilities of the outcome of
measuring the left qubit.


\begin{itemize}
\item $( \ket{00}+\ket{11} )/\sqrt2$ \ANSWER{......}
\item $( \ket{00}-\ket{11} )/\sqrt2$ \ANSWER{......}
\item $( \ket{01}+\ket{10} )/\sqrt2$ \ANSWER{......}
\item $( \ket{01}-\ket{10} )/\sqrt2$ \ANSWER{......}
\end{itemize}

\section{Blah 4}

\end{document}

%%% Local Variables:
%%% mode: latex
%%% fill-column: 115
%%% End:
